% Define o tipo de documento
\documentclass[landscape]{slides}

% Importa os caracteres latinos
\usepackage[latin1]{inputenc}
\usepackage[T1]{fontenc}
\usepackage{eurosym}

% Importa os pacotes para inclusao de graficos, links e v�deos
\usepackage{graphicx}
\usepackage{epsfig,color}
\usepackage{hyperref}
\usepackage{multimedia}

% Define o autor
\author{\url{https://ansol.org}}

% Define o titulo
\title{ANSOL\\Associa��o Nacional para o Software Livre}

% Define a data
\date{2018}

% Da inicio ao documento
\begin{document}

% TITLE
\begin{slide}
  \includegraphics{moita.png}

  \includegraphics{title-slide.png}
\end{slide}

% Participantes
\begin{slide}
  \begin{small}
    \begin{itemize}
      \item{} Marcos Marado \\ ANSOL - Associa��o Nacional para o Software Livre
      \item{} Gustavo Homem \\ ESOP - Associa��o de Empresas de Software Open Source Portuguesas
      \item{} Fernanda Ledesma \\ ANPRI - Associa��o Nacional de Professores de Inform�tica
      \item{} Manuela Apar�cio \\ ISCT-IUL MOSS - Mestrado Open Source
      \item{} Tiago Charters Azevedo \\ Livre
      \item{} Rui Lopo \\ CDU - Coliga��o Democr�tica Unit�ria
      \item{} F�tima Damaso \\ PAN - Pessoas-Animais-Natureza 
      \item{} BE - Bloco de Esquerda
    \end{itemize}
  \end{small}
\end{slide}

% O conte�do propriamente dito sobre a ANSOL (e SL no Estado)
\input{ANSOL.tex}

%% %% QUEST�ES
%% \begin{slide}
%% \center{\Huge{QUEST�ES?}}
%% 
%% \emph{http://ansol.org} \\
%% \emph{http://fsfe.org} \\
%% \emph{http://fsf.org} \\
%% \emph{https://github.com/marado/ANSOL-presentation} \\
%% \emph{http://listas.ansol.org/mailman/listinfo/ansol-geral} \\
%% \emph{https://savecodeshare.eu} \\
%% \emph{https://publiccode.eu}
%% \end{slide}

\end{document}
